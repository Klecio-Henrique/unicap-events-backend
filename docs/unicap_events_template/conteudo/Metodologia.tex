\chapter{Metodologia}


\section{Descrição Geral}

O projeto consiste na concepção e implementação de uma nova plataforma de eventos para o Instituto Humanitas da Universidade, com o objetivo de modernizar e aprimorar significativamente o sistema atual. A plataforma será um ponto central para a divulgação e gerenciamento de eventos acadêmicos, culturais e científicos promovidos pelo instituto, proporcionando uma experiência mais intuitiva, atrativa e eficiente para toda a comunidade universitária.

O design da plataforma será cuidadosamente elaborado para refletir a identidade visual do Instituto Humanitas, garantindo uma interface moderna, limpa e atraente. Além disso, serão integrados recursos interativos, como calendário, notificações automáticas e estatísticas, para aumentar o engajamento e a participação da comunidade nos eventos.


\section{Especificações Técnicas}

\subsection{Banco de Dados}


\subsection{Front-End}


\subsection{Back-End}


\section{Descrição de Interfaces}


\section{Casos de uso}


\chapter{Diagrama de Classes}


\section{Descrição do Diagrama}


\section{Elementos do Diagrama}
