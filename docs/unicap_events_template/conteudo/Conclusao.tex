\chapter{Conclusão}

\section{Considerações finais}

Ao concluir este projeto, destaca-se a importância da criação da plataforma de eventos acadêmicos Unicap Events para o Instituto Humanitas. A plataforma visa centralizar e facilitar o gerenciamento de eventos acadêmicos, proporcionando uma experiência mais eficiente para organizadores e participantes.

Analisando o contexto apresentado, identifica-se uma clara necessidade de aprimoramento do processo atual, visando uma gestão mais organizada e simplificada. Há uma demanda por melhorias na inserção, consulta e armazenamento de dados, bem como na otimização do cadastro e gerenciamento de eventos.

Assim, propõe-se a criação de uma plataforma que permita o cadastro e o gerenciamento dos eventos, preenchimento de formulários e visualização de informações, além de incluir recursos administrativos e de controle de processos. O objetivo é promover o acesso aos eventos acadêmicos, contribuindo para a igualdade de oportunidades e o desenvolvimento acadêmico.

Em suma, a criação desta plataforma de eventos acadêmicos representa um passo importante para a Universidade, oferecendo uma solução moderna, intuitiva e satisfatória para a gestão de eventos, e demonstra o compromisso da instituição com a inovação e a excelência acadêmica.